\documentclass[10pt,journal,compsoc]{IEEEtran}

% *** GRAPHICS RELATED PACKAGES ***
%
\ifCLASSINFOpdf
  \usepackage[pdftex]{graphicx}
  % declare the path(s) where your graphic files are
  \graphicspath{{../pdf/}{../jpeg/}}

  \DeclareGraphicsExtensions{.pdf,.jpeg,.png}
\else
  % or other class option (dvipsone, dvipdf, if not using dvips). graphicx
  % will default to the driver specified in the system graphics.cfg if no
  % driver is specified.
  % \usepackage[dvips]{graphicx}
  % declare the path(s) where your graphic files are
  % \graphicspath{{../eps/}}
  % and their extensions so you won't have to specify these with
  % every instance of \includegraphics
  % \DeclareGraphicsExtensions{.eps}
\fi


\usepackage[utf8]{inputenc}
\usepackage[portuguese]{babel}
\usepackage{amsfonts}
\usepackage{amssymb}
\usepackage{makeidx}
\usepackage[pdftex]{hyperref}
\usepackage{indentfirst}
\usepackage{pdfpages}
\usepackage{enumerate}
\usepackage{amsmath}
\usepackage{algorithmic}
\usepackage{array}
\usepackage{url}
\usepackage{cite}
\usepackage[nottoc]{tocbibind}
\usepackage{lipsum}
\usepackage{booktabs}

% correct bad hyphenation here
\hyphenation{op-tical net-works semi-conduc-tor}


\renewcommand\IEEEkeywordsname{Palavras-chave}

\begin{document}

\title{UM APLICATIVO CONTADOR DE MOEDAS UTILIZANDO OPENCV}

\author{\IEEEauthorblockN{Denis Ricardo da Silva Medeiros}
\IEEEauthorblockA{Departamento de Engenharia de Computação\\
		Universidade Federal do Rio Grande do Norte\\
		Natal, Rio Grande do Norte (84) 3342-2231\\
		Email: dnsricardo@gmail.com}
	\and \\
	\IEEEauthorblockN{Pedro Henrique de Medeiros Leite}
	\IEEEauthorblockA{Departamento de Enegenharia Elétrica\\
		Natal, Rio Grande do Norte (84) 3215-3731\\
		Email: pedrohenriquedemedeiros@gmail.com}
}

% make the title area
\maketitle

\begin{abstract}
Resumo vai aqui...
\end{abstract}

\begin{IEEEkeywords}
Processamento digital de imagem, moedas, classificação, contagem, redes neurais artificiais.
\end{IEEEkeywords}


% For peer review papers, you can put extra information on the cover
% page as needed:
% \ifCLASSOPTIONpeerreview
% \begin{center} \bfseries EDICS Category: 3-BBND \end{center}
% \fi
%
% For peerreview papers, this IEEEtran command inserts a page break and
% creates the second title. It will be ignored for other modes.
\section{Introdução}
\label{sec:introducao}

No dia a dia do consumidor, moedas são formas indispensáveis de se fazer pequenas compras, dar ou facilitar o troco em supermercados ou até mesmo para troca em cédulas. Dessa forma, devido ao comércio estar equipado com máquinas para o pagamento ser feito em cartão de crédito, no caso de grandes compras, ou contadores de cédulas em bancos para contagem de papel moeda, a contagem de moedas de forma automatizada para fins comerciais  muitas vezes acaba por ser negligenciado pelas pessoas.

Nesse contexto, a automatização desse processo poderia propocionar economia de tempo e consequentemente de dinheiro para as empresas, já que contar muitas moedas pode ser um trabalho árduo e demorado. Atualmente, já existem algumas tecnologias capazes de realizar tal tarefa, em geral atuando na comparação do peso das moedas. Entretanto, tais equipamentos ainda são bastante caros e inacessíveis para pequenos e microempresários.

Com base nesse cenário, uma possibilidade para realizar o processo de contagem de moedas de uma forma simples e barata que través de imagens capturadas por câmeras digitais, dispositivos amplamente acessíveis e presentes em telefones celulares, computadores, televisores, dentre outros. Contudo, o processamento digital de imagens contendo moedas não é simples e tem sido objeto de estudo de vários pesquisadores.

Em seu trabalho, \cite{bremananth2005new} utilizou como estratégia reconhecer os caracteres numéricos presentes nas moedas indianas como forma de identificá-las. Embora interessante, essa estratégia é falha paras as moedas de Real do Brasil, visto que um dos lados dela não possui número. Em outro trabalho, \cite{chetan2013robust} tentou realizar a identificação através de combinação de caracteríticas (\textit{feature matching}), tanto com as bordas quanto com o raio das moedas. Porém, pode não ser interessante para as moedas de Real, pois elas possuem tamanhos muito parecidos, variando o raio, em alguns casos, em somente 1 mm, como pode ser visto em \cite{bcb}.

Muitos dos trabalhos publicados nessa área abordam diferentes estratégias a respeito de que características das moedas serão utilizadas em sua indentificação. Porém, em geral, a maioria deles utiliza como métedo de classificação técnicas de inteligência artificial, mais especificamente redes neurais artificiais, como podem ser visto em \cite{bremananth2005new}, \cite{kaur2015coin}, \cite{modi2013automated} e até em trabalhos mais antigo, como e \cite{fukumi1992rotation}.

Após essa contextualização, este trabalho propõe uma nova estratégia na identificação e classificação de moedas, através de um aplicativo contador moedas de Real presentes em uma imagens obtidas por câmera digital. A ideia é que o usuário do aplicativo tire uma foto de um conjunto de moedas em uma cena padronizada, como com o fundo todo branco, por exemplo, e que ele informe ao usuário quantos reais estão presentes ali. Além do próprio aplicativo contador, também serão densenvolvidos módulos auxiliares para realizar a calibração do sistema, isto é, para treinar e validar a rede neural articial.

A tecnologia da classificação e identificação das moedas utilizada também será redes neurais artificiais, mas, diferente de outros trabalhos, as características a serem extraídas das moedas brasileiras tentarão tirar proveito do tamanho, das cores e de suas texturas. Por fim, como este projeto é apenas um protótipo, o aplicativo inicial será configurado para contar apenas moedas de R\$ 0,25, R\$  0,50 e R\$ 1,00.

\section{Metodologia}
\label{sec:metologia}

\begin{table*}[]
\centering
\caption{Detalhes das moedas de Real brasileiro}
\label{tab:moedas}
\begin{tabular}{@{}ccll@{}}
\toprule
\textbf{Valor Facial (R\$)} & \textbf{Diâmetro (mm)} & \multicolumn{1}{l}{\textbf{Bordo}} & \multicolumn{1}{l}{\textbf{Material}}                    \\ \midrule
0,01                        & 17,00                  & liso                               & Aço revestidode cobre                                    \\
0,05                        & 22,00                  & liso                               & Aço revestidode cobre                                    \\
0,10                        & 20,00                  & serrilhado                         & Aço revestidode bronze                                   \\
0,25                        & 25,00                  & serrilhado                         & Aço revestidode bronze                                   \\
0,50                        & 23,00                  & legenda                            & Aço inoxidável                                           \\
1,00                        & 27,00                  & serrilhaintermitente               & Aço inoxidável (núcleo) e aço revestido de bronze (anel) \\ \bottomrule
\end{tabular}
\end{table*}


A primeira etapa deste trabalho foi analisar os objetos de estudo, isto é, as moedas de Real do Brasil para decidir a melhor estratégia para identificá-las. Notou-se que elas possuem tamanhos diferentes, mas muito próximos, o que torna essa informação isolada muito sensível a erros. Notou-se, também, que elas possuem cores e texturas diferentes, dependendo da combinação de materiais com que elas são feitas. Por exemplo, a diferença  de raio da moeda R\$ 0,05 para a de R\$ 0,10 é de apenas 1 mm, conforme pode ser visto na Tabela \ref{tab:moedas}, com dados do Banco Central do Brasil \cite{bcb}.

A partir dessas informações, decidiu-se que seriam utilizadas informações sobre o tamanho da imagem, para tirar proveito da diferença do diâmetro, e do material de construção, que influencia tanto na cor e na textura das moedas presentes nas imagens. 





\section{Resultados}
\label{sec:resultados}

Resultados

\section{Conclusão}
\label{sec:conclusao}

Conclusão...

\medskip

\bibliographystyle{ieeetr}
\bibliography{references}

\end{document}


