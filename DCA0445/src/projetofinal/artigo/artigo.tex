% IEEEtran V1.7 and later provides for these CLASSINPUT macros to allow the
% user to reprogram some IEEEtran.cls defaults if needed. These settings
% override the internal defaults of IEEEtran.cls regardless of which class
% options are used. Do not use these unless you have good reason to do so as
% they can result in nonIEEE compliant documents. User beware. ;)
%
%\newcommand{\CLASSINPUTbaselinestretch}{1.0} % baselinestretch
%\newcommand{\CLASSINPUTinnersidemargin}{1in} % inner side margin
%\newcommand{\CLASSINPUToutersidemargin}{1in} % outer side margin
%\newcommand{\CLASSINPUTtoptextmargin}{1in}   % top text margin
%\newcommand{\CLASSINPUTbottomtextmargin}{1in}% bottom text margin



\documentclass[10pt,journal,compsoc]{IEEEtran}
% If IEEEtran.cls has not been installed into the LaTeX system files,
% manually specify the path to it like:
% \documentclass[10pt,journal,compsoc]{../sty/IEEEtran}


% For Computer Society journals, IEEEtran defaults to the use of 
% Palatino/Palladio as is done in IEEE Computer Society journals.
% To go back to Times Roman, you can use this code:
%\renewcommand{\rmdefault}{ptm}\selectfont


% *** CITATION PACKAGES ***
%
\ifCLASSOPTIONcompsoc
  % The IEEE Computer Society needs nocompress option
  % requires cite.sty v4.0 or later (November 2003)
  \usepackage[nocompress]{cite}
\else
  % normal IEEE
  \usepackage{cite}
\fi

% *** GRAPHICS RELATED PACKAGES ***
%
\ifCLASSINFOpdf
  \usepackage[pdftex]{graphicx}
  % declare the path(s) where your graphic files are
  \graphicspath{{../pdf/}{../jpeg/}}

  \DeclareGraphicsExtensions{.pdf,.jpeg,.png}
\else
  % or other class option (dvipsone, dvipdf, if not using dvips). graphicx
  % will default to the driver specified in the system graphics.cfg if no
  % driver is specified.
  % \usepackage[dvips]{graphicx}
  % declare the path(s) where your graphic files are
  % \graphicspath{{../eps/}}
  % and their extensions so you won't have to specify these with
  % every instance of \includegraphics
  % \DeclareGraphicsExtensions{.eps}
\fi


\usepackage[utf8]{inputenc}
\usepackage[portuguese]{babel}
\usepackage{amsfonts}
\usepackage{amssymb}
\usepackage{makeidx}
\usepackage[pdftex]{hyperref}
\usepackage{indentfirst}
\usepackage{pdfpages}
\usepackage{enumerate}
\usepackage{amsmath}
\usepackage{algorithmic}
\usepackage{array}
\usepackage{url}
\usepackage{cite}
\usepackage[nottoc]{tocbibind}

% correct bad hyphenation here
\hyphenation{op-tical net-works semi-conduc-tor}

\begin{document}

\title{UM APLICATIVO CONTADOR DE MOEDAS UTILIZANDO OPENCV}

\author{\IEEEauthorblockN{Denis Ricardo da Silva Medeiros}
\IEEEauthorblockA{Departamento de Engenharia de Computação\\
		Universidade Federal do Rio Grande do Norte\\
		Natal, Rio Grande do Norte (84) 3342-2231\\
		Email: dnsricardo@gmail.com}
	\and \\
	\IEEEauthorblockN{Pedro Henrique de Medeiros Leite}
	\IEEEauthorblockA{Departamento de Enegenharia Elétrica\\
		Natal, Rio Grande do Norte (84) 3215-3731\\
		Email: pedrohenriquedemedeiros@gmail.com}
}

% make the title area
\maketitle

\begin{abstract}
Resumo vai aqui...
\end{abstract}

\begin{IEEEkeywords}
Processamento digital de imagem, moedas, classificação, contagem, redes neurais artificiais.
\end{IEEEkeywords}


% For peer review papers, you can put extra information on the cover
% page as needed:
% \ifCLASSOPTIONpeerreview
% \begin{center} \bfseries EDICS Category: 3-BBND \end{center}
% \fi
%
% For peerreview papers, this IEEEtran command inserts a page break and
% creates the second title. It will be ignored for other modes.
\section{Introdução}
\label{sec:introduction}


No dia a dia do consumidor, moedas são formas indispensáveis de se fazer pequenas compras, dar ou facilitar o troco em supermercados ou até mesmo para troca em cédulas. Dessa forma, devido ao comércio estar equipado com máquinas para o pagamento ser feito em cartão de crédito, no caso de grandes compras, ou contadores de cédulas em bancos para contagem de papel moeda, a contagem de moedas de forma automatizada para fins comerciais  muitas vezes acaba por ser negligenciado pelas pessoas.

Nesse contexto, a automatização desse processo poderia propocionar economia de tempo e consequentemente de dinheiro para as empresas, já que contar muitas moedas pode ser um trabalho árduo e demorado. Atualmente, já existem algumas tecnologias capazes de realizar tal tarefa, em geral atuando na comparação do peso das moedas. Entretanto, tais equipamentos ainda são bastante caros e inacessíveis para pequenos e microempresários.

Com base nesse cenário, uma possibilidade para realizar o processo de contagem de moedas de uma forma simples e barata que través de imagens capturadas por câmeras digitais, dispositivos amplamente acessíveis e presentes em telefones celulares, computadores, televisores, dentre outros. Contudo, o processamento digital de imagens contendo moedas não é simples e tem sido objeto de estudo de vários pesquisadores.

Em seu trabalho, \cite{bremananth2005new} utilizou como estratégia reconhecer os caracteres numéricos presentes nas moedas indianas como forma de identificá-las. Embora interessante, essa estratégia é falha paras as moedas de Real do Brasil, visto que um dos lados dela não possui número. Em outro trabalho, \cite{chetan2013robust} tentou realizar a identificação através de combinação de caracteríticas (\textit{feature matching}), tanto com as bordas quanto com o raio das moedas. Porém, pode não ser interessante para as moedas de Real, pois elas possuem tamanhos muito parecidos, variando o raio, em alguns casos, em somente 1 mm, como pode ser visto em \cite{bcb}.

Muitos dos trabalhos publicados nessa área abordam diferentes estratégias a respeito de que características das moedas serão utilizadas em sua indentificação. Porém, em geral, a maioria deles utiliza como métedo de classificação técnicas de inteligência artificial, mais especificamente redes neurais artificiais, como podem ser visto em \cite{bremananth2005new}, \cite{kaur2015coin}, \cite{modi2013automated} e até em trabalhos mais antigo, como e \cite{fukumi1992rotation}.

Após essa contextualização, este trabalho propõe uma nova estratégia na identificação e classificação de moedas, através de um aplicativo contador moedas de Real presentes em uma imagens obtidas por câmera digital. A ideia é que o usuário do aplicativo tire uma foto de um conjunto de moedas em uma cena padronizada, como com o fundo todo branco, por exemplo, e que ele informe ao usuário quantos reais estão presentes ali. Além do próprio aplicativo contador, também serão densenvolvidos módulos auxiliares para realizar a calibração do sistema, isto é, para treinar e validar a rede neural articial.

A tecnologia da classificação e identificação das moedas utilizada também será redes neurais artificiais, mas, diferente de outros trabalhos, as características a serem extraídas das moedas brasileiras tentarão tirar proveito do tamanho, das cores e de suas texturas. Por fim, como este projeto é apenas um protótipo, o aplicativo inicial será configurado para contar apenas moedas de R\$ 0,25, R\$  0,50 e R\$ 1,00.

\section{Conclusion}
The conclusion goes here.

\appendices
\section{Proof of the First Zonklar Equation}
Appendix one text goes here.

% you can choose not to have a title for an appendix
% if you want by leaving the argument blank
\section{}
Appendix two text goes here.


% use section* for acknowledgment
\ifCLASSOPTIONcompsoc
  % The Computer Society usually uses the plural form
  \section*{Acknowledgments}
\else
  % regular IEEE prefers the singular form
  \section*{Acknowledgment}
\fi


The authors would like to thank...


% Can use something like this to put references on a page
% by themselves when using endfloat and the captionsoff option.
\ifCLASSOPTIONcaptionsoff
  \newpage
\fi

\medskip

\bibliographystyle{ieeetr}
\bibliography{references}



% <OR> manually copy in the resultant .bbl file
% set second argument of \begin to the number of references
% (used to reserve space for the reference number labels box)
%\begin{thebibliography}{1}

%\bibitem{IEEEhowto:kopka}
%H.~Kopka and P.~W. Daly, \emph{A Guide to {\LaTeX}}, 3rd~ed.\hskip 1em plus
% 0.5em minus 0.4em\relax Harlow, England: Addison-Wesley, 1999.

%\end{thebibliography}

% that's all folks
\end{document}


